\documentclass{article}
\usepackage{amsmath}
\usepackage{mathtools}
\usepackage{txfonts}
\usepackage[margin=0.5in]{geometry}

\setlength{\parskip}{\baselineskip}%
\setlength{\parindent}{0pt}%

\title{Homework 2}
\author{Daniel Hartig}


\begin{document}
\maketitle

\subsection*{Problem 1}

For probability function $$f_{T_n}(t) = (8n-8n^2t)\, I_{(1/2n, 1/n)}(t)$$ the distribution function over the same domain is obtained by integration $$F_{T_n}(t) = (8nt - 4n^2t^2 + C)\, I_{(1/2n, 1/n)}(t).$$ To be a valid distrubition function, $F$ must equal 0 at the lower bound of the domain, $t = 1/2n$, and equal 1 at the upper bound $t = 1/n$. At theses bounds, $F$ evaluates to 
\[\begin{aligned}
F_{T_n}(1/2n) &= \frac{8n}{2n} - \frac{4n^2}{(2n)^2} + C &= 3 + C \\
F_{T_n}(1/n)&= \frac{8n}{n} - \frac{4n^2}{n^2} + C &= 4 + C.
\end{aligned}\]
Therefore $C = -3$, $F(t) = 0$ for $ t<1/2n$ and $F(t) = 1$ for $ t > 1/n$, and \[F_{T_n}(t) = (8nt - 4n^2t^2 -3)\, I_{(1/2n, 1/n)}(t)\] meets the requirements of a distrubution function for all $n$. As $n$ approaches infinity, the range of $I_{(1/2n, 1/n)}(t)$ approaches the point $0$. We have already defined the $F(t) = 1$ for $ t > 1/n$, so we can say that \[\lim_{n\to\infty} F_{T_n}(t) = 1\quad \forall t > 0.\] Thus, the limiting distribution of $F_{T_n}(t)$ approaches $F(t)$, where $F(t)$ is 
\[F(t) = \begin{cases}
0, &t < 0 \\
1, &t \geq 0 \end{cases}\] The probability function of the limiting distribution for the sequence $T_1, T_2, ...$ is the derivative of the distribution function, and is 
\[f(t) = \begin{cases}
1, &t = 0 \\
0, &\text{otherwise}
\end{cases}\]

\iffalse
As $n\to\infty$, the mass function \[f_{T_n}(t) = (8n-8n^2t)\, I_{(1/2n, 1/n)}(t)\] approaches different values in its upper and lower bounds in $t$. At $t = 1/n$, \[\lim_{n\to\infty} (8n-8n^2t) = \lim_{n\to\infty} 8n-\frac{8n^2}{n} = 0.\] while at $t=1/2n$, \[\lim_{n\to\infty} (8n-8n^2t) = \lim_{n\to\infty} 8n-\frac{8n^2}{2n} = \lim_{n\to\infty} 4n = \infty.\] Meanwhile the domain of $t$ becomes $(1/2n, 1/n) = (0, 0)$. Thus the limiting distribution for $f_{T_n}$ is a vertical line at $t=0$, where the image of $f_{T_n}$ is all non-negative real numbers. 
\fi
\subsection*{Problem 2}
For $f_{X_n}(x)$, 
\iffalse
\[f_{X_n}(x) = \begin{dcases}
\frac{1}{n}, &x = n^2, \\
\frac{n-1}{n}, &x = 0, \\
0, &\textrm{otherwise}. 
\end{dcases}\]
\fi
the moment-generating function is calculated by \[M_X(t) = \sum_{x\in A}e^{tx}p(x) = e^{t\cdot 0}\left(\frac{n-1}{n}\right) + e^{tn^2}\left(\frac{1}{n}\right) = \frac{1}{n}\left(e^{tn^2} + n - 1\right).\] To calculate the $E(X_n)$ we need the first $t$ derivative of $M_X$, \[M'_X(t) = \frac{d}{dt}\left[\frac{1}{n}\left(e^{tn^2} + n - 1\right)\right] = ne^{tn^2},\] so that \[E(X_n) = M'_X(0) = ne^{0\cdot n^2} = n.\]
Thus,
\[\begin{aligned}
T_n &= X_n - E(X_n) \\
&= X_n - n
\end{aligned}\]
The probility distribution of $T_n$ can then be defined as 
\[f_{T_n}(x) = \begin{cases}
\frac{1}{n}, &x = n^2 - n \\
\frac{n-1}{n}, &x = -n \\
0, &\text{otherwise}
\end{cases}\]
The cumulative distribution function is 
\[F_{T_n}(x) = \begin{cases}
0, & x < -n \\
\frac{n-1}{n}, & -n \leq x < n^2 - n \\
1, &x \geq n^2 - n.
\end{cases}\]
As $n\to\infty$, the term $(n-1)/n$ approaches 1, so
\[\lim_{n\to\infty} F_{T_n} = \begin{cases}
0, &x < -n \\
1, &x \geq -n. \end{cases}\]
However, $-n$ iteself goes to $-\infty$, so $F_{T_n}$ does not converge. Since the distribution function of $T_n$ does not converge, a limiting distrubtion for the sequence $T_1, T_2, ...$ does not exist.

\iffalse
For $T_n = X_n - E(X_n)$, \[\lim_{n\to\infty} T_n = \lim_{n\to\infty} X_n - \lim_{n\to\infty} E(X_n).\] $E(X) = n$ therefore $E(X)$ approaches infinity. For $X_n$, the probability that $x=n^2$ is $1/n$ and approaches zero. The probability that $x=0$ approaches $1$. Then  \[\lim_{n\to\infty} T_n = \lim_{n\to\infty} X_n - \lim_{n\to\infty} E(X_n) = 0 - \infty = -\infty.\] Therefore, there is no limiting distribution for $f_X$. 
\fi

\subsection*{Problem 3}
\subsubsection*{Part a}
For the continuious probability function $f_X(x) = 3(1-x)^2\,I_{(0,1)}(x),$ the cumulative distribution function is obtained by integration \[F_X(x) = \int 3(1-x)^2\,I_{(0,1)}(x)\,dx = (x-1)^3 + C.\] For the bounds of the probability function $0$ and $1$, the values of $F_X$ are 
\[\begin{aligned}
(x-1)^3 + C\,\big|_0 &= -1 + C \\
(x-1)^3 + C\,\big|_1 &= C.
\end{aligned}\]
Set $C = 1$, $F_X = 0$ for $x < 0$ and $F_X = 1$ for $x > 1$, and then 
\[F_X(x) = \begin{cases}
0, &x<0\\
(x-1)^3 + 1, & 0<x<1 \\
1, &x>1\end{cases}\] is a valid distribution function. Since the maxium possible value of the probability function is $x=1$, we expect $X_{(n)}$ to converge to 1 as more random variables are added to the sequence. Since we expect the sequence of maxima to converge to $1$, we can say that it will converge to a random variable $X=1$. To apply this to Definition 5.5.1, we solve for 
\[\begin{aligned}
P\left(\left|X_{(n)}-1\right|\geq\epsilon\right) &= P\left(X_{(n)}\geq 1+\epsilon\right) + P\left(X_{(n)} \leq 1 - \epsilon\right) \\
&=  P\left(X_{(n)} \leq 1 - \epsilon\right).
\end{aligned}\]
Using the distribution function for $X_i$, and a change of variables from $0 < x < 1$ to $0 < (1-\epsilon) < 1$, we can say that for all $X_i$ with $1\leq i\leq n$,
\[ P(X_i \leq 1 - \epsilon) = \left((1-\epsilon)-1\right)^3+1 = 1 - \epsilon^3.\]
Since the $X_i$ are independent of each other, the probability that all $X_i$ in the series $X_1, X_2, ... , X_n$ are less than $1-\epsilon$ is \[P\left(X_{(n)} < 1 - \epsilon\right) = \left(1-\epsilon^3\right)^n.\] This goes to zero for all $\epsilon > 0$, therefore the maximum is proven to converge to $1$. We can change variables again to $\epsilon = t/n^{1/3}$ where since $n \geq 1$, $t > 0$ to match ranges. We then get
\[\begin{aligned}
P\left(X_{(n)} \leq 1 - t/n^{1/3}\right) &= \left(1-\left(\frac{t}{n^{1/3}}\right)^3\right)^n = \left(1-\frac{t^3}{n}\right)^n \to \left(e^{-t^3}\right)\, I_{(0,\infty)}(t) \\
P\left(n^{1/3}\left(1-X_{(n)}\right)\leq t\right) &\to \left(1 - e^{-t^3}\right)\,I_{(0,\infty)}(t).
\end{aligned}\]
The left side of this expression is the distribution function for $T_n$. To find the probability function by Theorem 5.5.12, we must take the $t$ derivative of the distribution function
\[f_{T_n}(t) = \frac{d}{dt}\left[1 - e^{-t^3}\right] = 3\,t^2\,e^{-t^3}\,I_{(0,\infty)}(t).\]
Therefore, as $n$ goes to infinity, the sequence $T_1, T_2, ...$ converges to the above probability function. 

\subsubsection*{Part b}
By Theorem 5.5.4, if $X_1, X_2, ...$ converges in probability to $X$, and $h$ is a continuous function, then $h(X_1), h(X_2), ...$ converges in probability to $h(X)$. Let $h(y) = \sqrt{y}$ and 
\[V_n = h(T_n) = \sqrt{(T_n)}.\] Since we know that $T_n$ converges to 
\[3\,t^2\,e^{-t^3}\,I_{(0,\infty)}(t),\] then $V_n$ converges to 
\[\sqrt{ 3\,t^2\,e^{-t^3}}\,I_{(0,\infty)}(t) = \sqrt{3}\,t\exp{\left(\frac{-t^3}{2}\right)}\,I_{(0,\infty)}(t)\]




\iffalse
Since the $X_i$ are independent of each other, the probability that all $X_i$ in the series $X_1, X_2, ... , X_n$ are less than $x$ is \[P\left(\text{max}\{X_1, ... , X_n\} < x\right) = \left((x-1)^3 + 1\right)^n.\]
This is the definition of a distribution function 
\[F_{\text{max}}(x) = \begin{cases}
0, &x < 0 \\
\left((x-1)^3 + 1\right)^n, &0\leq x \leq1 \\
1, & x > 1.\end{cases}\] The limit of this distribution in $n$ is 
\[F_{\text{max}}(x) = \begin{cases}
0, &x < 1 \\1, & x \geq 1.\end{cases}\]
\fi

\iffalse
, we can determine the probability that any one $_i$ is more than $\epsilon$ away from the maximum possible value, $1$. We can say that for all $X_i$ with $0<i<n$,
\[ P(X_i \leq 1 - \epsilon) = \left((1-\epsilon)-1\right)^3+1 = 1 - \epsilon^3\] for $x = 1 - \epsilon; \,\,0 \leq \epsilon \leq 1$. Since the $X_i$ are independent of each other, the probability that all $X_i$ in the series $X_1, X_2, ... , X_n$ are less than $1-\epsilon$ is \[P\left(\text{max}\{X_1, ... , X_n\} < 1 - \epsilon\right) = \left(1-\epsilon^3\right)^n.\]
This is the definition of a distribution function \[F_{\text{max}}(1-\epsilon) = \left(1-\epsilon^3\right)^n\,I_\epsilon(0,1).\] The distribution function for $T_n$ can thus be obtained as 
\[T_n(1-\epsilon) = n^{1/3}\left(1-\left(1-\epsilon^3\right)^n\right)\,I_\epsilon(0,1).\]
\fi

\iffalse
We can now use Theorem 5.4.4 from the textbook to find an expression for the probability function of $\max\{X_1, ...\,, X_n\}$, the $n$-th order statistic of $X$:
\[\begin{aligned}
f_{X_{(n)}}(x) &= \frac{n!}{(n-1)!(n-n)!}f_X(x)\left[F_X(x)\right]^{n-1}\left[1-F_X(x)\right]^{n-n} \\
&= n\,f_X(x)\left[F_X(x)\right]^{n-1} \\
&= n\left(3(1-x)^2\right)\left[(x-1)^3 + 1\right]^{n-1}I_{(0,1)}(x)
\end{aligned}\]

On the domain $0<x<1$. We can find an expression for \[T_n = n^{1/3}\left(1-n\,f_X(x)\left[F_X(x)\right]^{n-1}.\right)\] While considering the limit with respect to $n$ of this function, it is important to note that $F_X \leq 1$ for all $x$ in the domain since it is a distribution function; therefore, $\left[F_X(x)\right]^{n-1}$ tends to zero as $n$ tends to infinity.
\fi

\subsection*{Problem 4}
The moment generating function of the sum of independent random variables in the sequence $X_1, X_2, ...$ is 
\[M_{X_1}(t)M_{X_2}(t)... .\]
The moment generating function of a Poisson distribution is $\exp(\lambda(e^t-1))$. Thus the moment generating function for $T_n = X_1 + ... + X_n$ is
\[\begin{aligned}M_{T_n}(t) &= \prod_{k=1}^{n} \exp\left(k^{-2}\left(e^t - 1\right)\right) \\
&= \exp\left(\left(e^t-1\right)\sum_{k=1}^{n} \frac{1}{k^2}\right).\end{aligned}\] 
This is itself the moment generating function of a Poisson distrubution with \[\lambda = \sum_{k=1}^{n} \frac{1}{k^2}.\]
As $n$ goes to infinity, the sum of this infinite series is $\pi^2/6$. Therefore, as $n$ goes to infinity, the sequence $T_1, T_2, ...$ converges to a limiting distribution of Poisson$\,\left(\pi^2/6\right)$, the probability mass function of which is 
\[\frac{\left(\frac{\pi^2}{6}\right)^je^{-\pi^2/6}}{j!} =  \frac{\pi^{2j} e^{-\pi^2/6}}{6^j j!}\]




\iffalse
The probability function of a Poisson random variable with $E(x) = \lambda$ is \[\frac{\lambda^je^{-\lambda}}{j!},\] so for $X_k$ with $E(X_k) = k^{-2}$, \[f_{X_k}(x) = \frac{k^{-2j}\exp\left(-\frac{1}{k^2}\right)}{j!},\] for all non-negative integrers $j$, and $0$ otherwise.
%\iffalse
$T_n$ can then be expressed as the sum of a series 
\[\begin{aligned}
T_n &= \sum_{k=1}^{n}\frac{k^{-2j}\exp\left(-\frac{1}{k^2}\right)}{j!} \\
&=\frac{1}{j!}\sum_{k=1}^{n}\, k^{-2j}e^{-\frac{1}{k^2}}.\end{aligned}\]
%\fi


Since $e^{-\frac{1}{k^2}} <= 1 $ for all integers $k > 0$, then $k^{-2j}e^{-\frac{1}{k^2}} \leq k^{-2j}$ as well. $k^{-2j}$ or $\frac{1}{k^{2j}}$ is a basic null sequence, so $k^{-2j}e^{-\frac{1}{k^2}}$ is also a null sequence. Since $k^{-2j}e^{-\frac{1}{k^2}}$ is a null sequence, then  $\sum_{k=1}^{n}\, k^{-2j}e^{-\frac{1}{k^2}}$ is a convergent series. Therefore the last entry in the sequence $T_n$, is itself a convergent series. 
\fi

\subsection*{Problem 5}
For iid randm variables $X_1, X_2, ...$, the sample mean $\bar{X}$ is $(1/n)\sum X_i$. In otherwords, 
\[\sum_{i=1}^{n}X_i = n\bar{X}.\]
Since $X_i$ is a Bernoulli variable, the only possible outcomes are $0$ and $1$. $0^2=0$ and $1^2=1$, $X_i^2 = X_i$, and \[\sum_{i=1}^n X_i^2 = \sum_{i=1}^n X_i = n\bar{X}.\] We can re-write the formula for $T_n$ as 
\[T_n = \sqrt{n}\left(\frac{4n\bar{X}-2n}{n\bar{X}}\right) = \sqrt{n}\left(\frac{\bar{X} - \frac{1}{2}}{\frac{1}{4}\bar{X}}\right).\]
By the Central Limit Theorem, this function has a limiting standard normal distribution with $\mu = 1/2$ and $\sigma = (1/4)\bar{X}$. We can find $\bar{X}$ by the Law of Large numbers
\[\lim_{n\to\infty}\bar{X_n} = \mu = p = \frac{1}{2}\] as given in the problem statement. Therefore, as $n\to\infty$, the sequence $T_1, T_2, ...$ converges to a limiting distribution of Normal$(1/2, 1/8)$. 

\iffalse
The sum of the distributions of multiple Bernoulli random variables is the binomial distribution. For Bernoulli random variables $X_i$ with mean $p = 1/2$, as $n$ goes to infinity, \[\sum_{i=1}^n X_i = nE(X_i) = n^2p = \frac{1}{2}n^2\] due to the Law of Large Numbers. Since $X_i$ is a Bernoulli variable, the only possible outcomes are $0$ and $1$. Since $0^2=0$ and $1^2=1$, $X_i^2 = X_i$, and \[\sum_{i=1}^n X_i^2 = \sum_{i=1}^n X_i = \frac{1}{2}n^2.\] Therefore, as $n\to\infty$, \[\lim_{n\to\infty}T_n = \lim_{n\to\infty}\sqrt{n}\left(\frac{4\sum_{i=1}^2 X_i - 2n}{\sum_{i=1}^n X_i^2}\right) =\lim_{n\to\infty} \sqrt{n}\left(\frac{2n^2-2n}{1/2n^2}\right) = \lim_{n\to\infty}4\left(\sqrt{n} - \frac{1}{\sqrt{n}}\right).\]
This limit goes to infinity. Therefore, as $n$ goes to infinity, the sequence  $T_1, T_2, ...$ does not converge. 
\fi

\subsection*{Problem 6}
\subsubsection*{Part a}
$X$ is a random variable with $E(X) = \theta$. Let $g(X) = 1/\sqrt{X}$ be an estimator for $1/\sqrt{\theta}$; $g'(X) = -1/2\,x^{-3/2}.$ The first order Taylor approximation to the mean of $1/\sqrt{X_n}$ is
\[\begin{aligned}
g(X) &= g(\theta)+g'(\theta)(X-\theta) \\
\frac{1}{\sqrt{X_n}} &= \frac{1}{\sqrt{\theta}} -\frac{1}{2}\theta^{-\frac{3}{2}}(X_n - \theta) \\
E\left[\frac{1}{\sqrt{X_n}}\right] &= \frac{1}{\sqrt{\theta}} + -\frac{1}{2}\theta^{-\frac{3}{2}}\,E\left[X_n - \theta\right] \\
\end{aligned}\]
Since $\theta$ is the mean of $X$, $E[X-\theta] = 0$, and so 
\[E\left[\frac{1}{\sqrt{X}}\right] = \frac{1}{\sqrt{\theta}}\]
\subsubsection*{Part b}
Since $g''(x) = 3/4x^{-5/2}$; the second order approximation to the mean of $1/\sqrt{X_n}$ is 
\[\begin{aligned}
g(X) &= g(\theta)+g'(\theta)(X-\theta)+\frac{1}{2}g''(\theta)(X_n-\theta)^2 \\
E\left[\frac{1}{\sqrt{X}}\right] &= \frac{1}{\sqrt{\theta}} + \frac{3}{4}\theta^{-\frac{5}{2}}E\left[(X_n - \theta)^2\right] \\
\end{aligned}\]
Now $\theta$ is the mean of $X$, $E\left[(X - \theta)^2\right]$ is the definition of the variance of $X$, which is $\theta^3$. Therefore
\[E\left[\frac{1}{\sqrt{X_n}}\right] = \frac{1}{\sqrt{\theta}} + \frac{3}{4}\sqrt{\theta}=\frac{1+\frac{3}{4}\theta}{\sqrt{\theta}}\] 
\subsubsection*{Part c}
The first order approximation to the variance of $1/\sqrt{X_n}$ is 
\[\begin{aligned}
\text{Var}\left[\frac{1}{\sqrt{X_n}}\right] &= \left(-\frac{1}{2}\theta^{-\frac{3}{2}}\right)^2\text{Var}\left[X_n\right] \\
&= \frac{1}{4}\frac{1}{\theta^3}\theta^3 = \frac{1}{4}
\end{aligned}\]
\subsection*{Problem 7}
To generate a sufficient statistic, we must factorize the joint probability density function
$f_X(\mathbf{x}|\theta)$ into two terms $g(T(\mathbf{x})|\theta)$ and $h(\mathbf{x})$ as 
\[\begin{aligned}
f_X(x) &= \prod_{k=1}^n(\theta+1)x_k^\theta I_{(0,1)}(x) \\
&= (\theta+1)^n \exp\left(\theta\sum_{k=1}^n \log{x_k}\right) I_{(0,1)}(x).\\
\end{aligned}\]
Therefore 
\[h(\mathbf{x}) = I_{(0,1)}(x)\]
and 
\[g(T(\mathbf{x})|\theta) = (\theta+1)^n \exp\left(\theta\sum_{k=1}^n \log{x_k}\right),\]
so 
\[T(\mathbf{x}) = \sum_{k=1}^n \log{x_k}\]
is a sufficient statistic for $\theta$ for every $x$ in the sample space. 

 





\end{document}