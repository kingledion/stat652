\documentclass{article}
\usepackage{amsmath}
\usepackage{mathtools}
\usepackage{txfonts}
\usepackage{fancyhdr}
\usepackage[margin=0.5in]{geometry}
\usepackage{graphicx}
\graphicspath{ {images/} }

\setlength{\parskip}{\baselineskip}%
\setlength{\parindent}{0pt}%
\pagestyle{fancy}
\fancyhf{}
\renewcommand{\headrulewidth}{0pt}

\begin{document}
\fancyhead[R]{\null\hfill\begin{tabular}[t]{l@{}}
	\textbf{Daniel Hartig} \\
	Stat 652 \\
	Homework 4
\end{tabular}}
\setlength{\headheight}{4\baselineskip}



\subsection*{Problem 1}
\subsubsection*{a.}
\begin{align*}
f(\mathbf{x}|\theta)&=\prod_{i=0}^n\left[\frac{1}{2\theta}\exp{\left(\frac{-\sqrt{x_i}}{\theta}\right)}I_{(0,\infty)}(x_i)\right] \\
&=\frac{1}{2^n\theta^{2n}}\exp{\left(\frac{-1}{\theta}\sum_{i=0}^n\sqrt{x_i}\right)}\prod_{i=0}^nI_{(0,\infty)}(x_i)
\end{align*}
Let 
\[
c(\theta)=\frac{1}{2^n\theta^{2n}}, \qquad h(x)=\prod_{i=0}^n I_{(0,\infty)}(x_i), \qquad w_1=\frac{-1}{\theta}, \qquad t = \sum_{i=0}^n\sqrt{x_i}
\]
\iffalse
Due to factorization (Theorem 6.2.6), $T(\mathbf{x}) = \sum_{i=0}^n\sqrt{x_i}$ is a sufficient statistic for $\theta$. 
\fi
Since
\begin{align*}
\left\{w_1(\theta):\theta \in \Theta\right\}\quad \rightarrow\quad \left\{\frac{-1}{\theta}:\theta > 0\right\}\quad \rightarrow\quad \left(-\infty, 0\right)
\end{align*} 
contains an open set on $\mathbb{R}^\prime$, $T(\mathbf{x}) = \sum_{i=0}^n\sqrt{x_i}$ is a complete sufficient statistic. The expectation of any $X$ can be calculated as 
\begin{align*}
E(X) &= \int_0^\infty x\frac{1}{2\theta^2}\exp{\left(\frac{-\sqrt{x_i}}{\theta}\right)} dx \\
&=\left.\frac{1}{2\theta^2}\exp{\left(\frac{-\sqrt{x}}{\theta}\right)}\left(-12\theta^4-12\theta^3\sqrt{x}-6\theta^2x-\theta x^{3/2}\right)\right|_0^\infty &\text{solved with Wolfram Alpha} \\
&=\left.\exp{\left(\frac{-\sqrt{x}}{\theta}\right)}\left(-6\theta^2-6\theta\sqrt{x}-3x-\frac{1}{2\theta}x^{3/2}\right)\right|_0^\infty \\
&= e^{-\infty}(...)-e^0\left(-6\theta^2-0\right) = 6\theta^2
\end{align*}
Using a similar derivation, the expectation of $\sqrt{X}$ is
\begin{align*}
E(\sqrt{X}) &= \int_0^\infty \sqrt{x}\frac{1}{2\theta^2}\exp{\left(\frac{-\sqrt{x_i}}{\theta}\right)} dx \\
&=\left.\exp{\left(\frac{-\sqrt{x}}{\theta}\right)}\left(-2\theta-2\sqrt{x}-\frac{1}{\theta}x\right)\right|_0^\infty &\text{solved with Wolfram Alpha} \\
&=2\theta
\end{align*}
To find an unbiased estimator, we investigate the expectation of $T(\mathbf{x})$,
\begin{align*}
E\left(T(\mathbf{x})\right) &= E\left(\sum_{i=0}^n\sqrt{x_i}\right) \\
&=\sum_{i=0}^n E\left(\sqrt{x_i}\right) \\
&=n(2\theta)
\end{align*}
Therefore, the expectation of $\hat{\theta} = T(\mathbf{x})/2n$, \[E(\hat{\theta}) = E\left(\frac{T}{2n}\right) = \theta\] shows that $\hat{\theta}$ is an unbiased estimator of $\theta$. Since it is also a function of a complete sufficient statistic of $\theta$, by Theorem 7.3.23, $\hat{\theta}$ is the unique best unbiased estimator of $\theta$.

\subsubsection*{b.}
The Carmer-Rao lower bound for the variance of $\hat{\theta}$ given that $\hat{\theta}$ is a function of $T(\mathbf{X})$ and $X_1, X_2, ...$ are iid variables is given by 
\[\frac{1}{nE_\theta\left(\left[\frac{d}{d\theta}\log f(X|\theta)\right]^2\right)}.\]
We resolve the denominator using values of expectation from part a, 
\begin{align*}
\frac{d}{d\theta}\log f(X|\theta) &= \frac{d}{d\theta}\left(-2\log(\sqrt{2}\theta)-\frac{\sqrt{x}}{\theta}\right) = \frac{-2}{\theta}+\frac{\sqrt{x}}{\theta^2} \\
\left(\frac{d}{d\theta}\log f(X|\theta)\right)^2 &= \frac{4}{\theta^2}-\frac{4\sqrt{x}}{\theta^3}+\frac{x}{\theta^4} \\
E\left[\left(\frac{d}{d\theta}\log f(X|\theta)\right)^2\right] &= E\left[\frac{4}{\theta^2}-\frac{4\sqrt{x}}{\theta^3}+\frac{x}{\theta^4}\right] = \frac{4}{\theta^2}-\frac{4E(\sqrt{x})}{\theta^3}+\frac{E(x)}{\theta^4} \\
&=\frac{4}{\theta^2} - \frac{8\theta}{\theta^3}+\frac{6\theta^2}{\theta^4} = \frac{2}{\theta^2}
\end{align*}
which gives us a Cramer-Rao lower bound of 
\[\frac{1}{n\cdot\frac{2}{\theta^2}} = \frac{\theta^2}{2n}.\]

\subsubsection*{c.}
\begin{align*}
\mathrm{Var}{(\hat{\theta})} &= \frac{1}{4n^2}\mathrm{Var}(T) = \frac{1}{4n^2}\mathrm{Var}\left(\sum_{i=0}^n \sqrt{X_i}\right) = \frac{1}{4n^2}\sum_{i=0}^n \mathrm{Var}(\sqrt{X_i})\\
&=\frac{n}{4n^2}\left(E(X)-\left[E(\sqrt{X})\right]^2\right) \\
&=\frac{1}{4n}\left[6\theta^2-(2\theta)^2\right] \\
&= \frac{\theta^2}{2n}
\end{align*}

\subsubsection*{d.}
We can reuse the complete, sufficient statistic $T(\mathbf{x})$ that we derived in part a. From part a we see that 
\[E\left(T(\mathbf{x})\right)=2n\theta.\]
Therefore, to find an unbiased estimator for $6\theta^2$, we investigate $T^2$ using information from part c,
\begin{align*}
E\left(T^2\right) &= \mathrm{Var}(T) + \left[E(T)\right]^2 \\
&=2n\theta^2 + (2n\theta)^2 \\
&=2n(2n+1)\theta^2
\end{align*}
Thus \[\widehat{6\theta^2} = \frac{3T^2}{n(2n+1)}; \qquad E\left(\widehat{6\theta^2}\right) = E\left(\frac{3T^2}{n(2n+1)}\right) = \frac{3E\left(T^2\right)}{n(2n+1)} = 6\theta^2\] 
is an unbiased estimator for $6\theta^2$. Since it is also a function of $T(\mathbf{x})$, and we have determined that this is a complete, sufficient statistic in part a, by Theorem 7.3.23, $\widehat{6\theta^2}$ is the unique best unbiased estimator of $6\theta^2$.

\subsubsection*{e.}
Let $\tau(\theta) = 6\theta^2$, so that $\tau^\prime(\theta) = 12\theta$. Then the Cramer-Rao lower bound is given by 
\begin{align*}
\frac{\left[\tau^\prime(\theta)\right]^2}{nI_1(\theta)} = \frac{144\theta^2}{n\cdot\frac{2}{\theta^2}} = \frac{72\theta^4}{n}
\end{align*}

\subsubsection*{f.}


\subsubsection*{g.}
The MME is obtained by setting $m_1=\bar{X}$ equal to $\mu_1^\prime = E(X)$. Since $E(X) = 6\theta^2$, we get $\widehat{6\theta^2} = \bar{X}$ as the MME estimator of $6\theta^2$. To find the variance we must take 
\begin{align*}
\mathrm{Var}(\bar{X}) = \mathrm{Var}\left(\frac{1}{n}\sum_{i=0}^n X_i\right) = \frac{1}{n^2}\sum_{i=0}^n \mathrm{Var}(X_i) = \frac{1}{n}\mathrm{Var}(X_i)
\end{align*}
The variance of $X$ (any of the iid $X_i$) can be calculated from its definition, using the values of $E(X)$ from part a, and
\begin{align*}
E(X^2) &= \int_0^\infty \frac{x^2}{2\theta^2}\exp\left(\frac{-\sqrt{x}}{\theta}\right)dx \\
&=120\theta^4 &\text{solved with Wolfram Alpha}
\end{align*}

\begin{align*}
\mathrm{Var}(\bar{X}) &= \frac{1}{n}\left(E(X^2) - \left[E(X)\right]^2\right) \\
&=\frac{1}{n}\left(120\theta^4-36\theta^4\right) = \frac{84\theta^4}{n} \\
\end{align*}



\end{document}